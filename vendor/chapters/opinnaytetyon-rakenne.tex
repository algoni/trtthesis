\chapter{Opinnäytetyön rakenne}

Tässä esimerkkejä luvuista ja aliluvuista.

\section{Lähdeluettelo}

Lähdeluettelo voi \LaTeX:ssa tehdä joko tietokantamuotoisesti
\textit{bibtex}-käskyn avulla tai vaihtoehtoisesti tässä esitetyllä tyylillä.
Lisätietoa löytyy esim. Wikipediasta ''\textit{bibtex}''-hakusanalla.

Tältä näyttää viittaus perinteisellä numeromerkinnällä \cite{Hirs,Mittelbach}.

% Huom.: oikeaoppinen tapa viitata olisi seuraava:
%
% ...numeromerkinnällä~\cite{Hirs}.
%
% Merkintä "~" toimii kuten välilyönti, mutta estää rivinvaihdon
% ja seuraavan rivin alkamisen rumasti viittauksella.

\subsection{Aliesimerkki}

Turhaa tekstiä esimerkin vuoksi.

\section{Matemaattisista merkinnöistä}

\LaTeX:lla yhtälöiden kirjoittaminen on helppoa. Seuraavassa ensin esimerkki
\begin{equation}
  \label{eq:fourier}
  G^+(t,t')= \int G^+(E) exp[-iE(t-t')/\hbar] dE,
\end{equation}
jonka jälkeen kappale jatkuu näin.
Tekstissä kaavat $G^+(t,t')= \int G^+(E) exp[-iE(t-t')/\hbar] dE$ näyttävät
tällaisilta. Kaavoihin (ks. \ref{eq:fourier}) on myös helppo viitata.

% Esimerkkikuva.
% Huomaa, että useimmat ohjelmat latovat PDF-kuvan
% A4-sivulle, jolloin kuvassa on mukana paljon tyhjää.
% Tämän voi poistaa PDF-editoreilla. Esim.
% PDF-X-Change-ohjelmistossa on toiminto "remove white spaces".
%
% Huomaa myös, että Latex sijoittaa kuvan aikaisintaan
% tähän kohtaan tekstissä. Yleensä kuvat kannattaa
% ladata n. sivun verran aikaisemmin kun niihin viitataan.
% Muuten seuraava sopiva paikka saattaa tulla vasta
% parin sivun päähän viittauksesta.

\centeredpicture{esimKuva}{Matlabilla tehty PDF-muotoinen esimerkkikuva.}

Latexiin voi tuoda kuvia useassa eri formaatissa.
Näppärintä on käyttää pdflatex-kääntäjää, jolloin
sallitut formaatit ovat \verb+*.png+, \verb+*.jpg+
sekä \verb+*.pdf+. Kuvassa~\ref{fig:esimKuva} on
PDF-formaatissa tuotu Matlabilla tehty
esimerkkikuva.
